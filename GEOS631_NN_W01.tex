\documentclass[12pt]{article}
\renewcommand{\familydefault}{\sfdefault}
\usepackage{GEOS631_homework}

%----------------------------------------------------------------------------
%	NAME AND CLASS SECTION
%----------------------------------------------------------------------------

\newcommand{\hmwkTitle}{Written Assignment\ \#1} % Assignment title
\newcommand{\hmwkDueDate}{Monday,\ September\ 17,\ 2012} % Due date
\newcommand{\hmwkClass}{GEOS F631} % Course/class
\newcommand{\hmwkClassTime}{} % Class/lecture time if you wish it added
\newcommand{\hmwkClassInstructor}{Freymueller, Pettit} % Teacher/lecturer
\newcommand{\hmwkAuthorName}{Namepart1 Namepart2} % Your name

%----------------------------------------------------------------------------
%	TITLE PAGE
%----------------------------------------------------------------------------

\title{
\vspace{1in}
\textmd{\textbf{\hmwkClass:\ \hmwkTitle}}\\
\vspace{0.1in}
\normalsize\textit{Instructors: \hmwkClassInstructor\ \hmwkClassTime}\\
\vspace{0.1in}\large{Due\ on\ \hmwkDueDate}\\
\vspace{1in}
}

\author{\textbf{\hmwkAuthorName}}
\date{} % Insert date here if you want it to appear below your name

%----------------------------------------------------------------------------

\begin{document}
\maketitle

%----------------------------------------------------------------------------
%	TABLE OF CONTENTS
%----------------------------------------------------------------------------

%\setcounter{tocdepth}{1} % Uncomment this line if you don't want subsections listed in the ToC
%\newpage % Uncomment this line if you like a page break between the Title and TOC
\vspace{0.5in}
\tableofcontents
\newpage

%----------------------------------------------------------------------------
%	PROBLEM 1
%----------------------------------------------------------------------------

% To have just one problem per page, simply put a \clearpage after each problem
\begin{homeworkProblem}[Problem \arabic{homeworkProblemCounter}: T \& S, 1--1 ]

\textit{If the area of the oceanic crust is $3.2 \times 10^8\ \mathrm{km}^2\ \mathrm{yr}^{-1}$ and new seafloor is now being created at the rate of $2.8\ \mathrm{km}^2\ \mathrm{yr}^{-1}$, what is the mean age of the oceanic crust? Assume that the rate of seafloor creation has been constant in the past.}

\problemAnswer{
The spreading has been going on for ...}

\end{homeworkProblem}


%----------------------------------------------------------------------------
%	PROBLEM 2
%----------------------------------------------------------------------------

\begin{homeworkProblem}[Problem \arabic{homeworkProblemCounter}: T \& S, 1--10 ]

\textit{Determine the velocity of seafloor spreading on the South East Indian Rise from the magnetic anomaly profile given in Figure 1--30b.}

\problemAnswer{
Solution.
}

\end{homeworkProblem}

%----------------------------------------------------------------------------
%	PROBLEM 3
%----------------------------------------------------------------------------

\begin{homeworkProblem}[Problem \arabic{homeworkProblemCounter}: T \& S, 1--17 ]

\textit{What is the spreading rate between the North American and Eurasian plates in Iceland ($65^{\circ}\mathrm{N}$, $20^{\circ}\mathrm{W}$)?}

\problemAnswer{
Solution.
}

\end{homeworkProblem}

%----------------------------------------------------------------------------
%	PROBLEM 4
%----------------------------------------------------------------------------

\begin{homeworkProblem}[Problem \arabic{homeworkProblemCounter}: T \& S, 1--18 ]

\textit{What is the relative plate velocity between the Nazca and South American plates at Lima ($12^{\circ}\mathrm{S}$, $77^{\circ}\mathrm{W}$)?}

\problemAnswer{
Solution.
}

\end{homeworkProblem}

%----------------------------------------------------------------------------
%	PROBLEM 6
%----------------------------------------------------------------------------

\addtocounter{homeworkProblemCounter}{1} % increment counter - there's no Problem 5
\begin{homeworkProblem}[Problem \arabic{homeworkProblemCounter}: T \& S, 1--21 ]

\textit{Show that a triple junction of three transform faults cannot exist.}

\problemAnswer{
Solution.
}

\end{homeworkProblem}

%----------------------------------------------------------------------------
%	PROBLEM 7
%----------------------------------------------------------------------------

\begin{homeworkProblem}

\textit{Given the plate configuration of Figure 1-36, show why equation 1-20 is true. Start with the angular velocities of the three plates being $\omega_A$, $\omega_B$, and $\omega_C$ in some reference frame, and $\omega_{BA} = \omega_B - \omega_A$. First show why $\omega_{BA} + \omega_{CB} + \omega_{CA} = 0$ always, and then derive 1--20. What is the key approximation (more specific than flat earth)? What conditions�might cause�the�approximation�to�break down?}

\problemAnswer{
Solution.
}

\end{homeworkProblem}

%----------------------------------------------------------------------------
%	PROBLEM 8
%----------------------------------------------------------------------------

\begin{homeworkProblem}
\textit{Show that all three of these ways of describing a plate rotation are equivalent: pole of rotation and angular speed, angular velocity vector describing a rotation about a geocentric axis, or rotation angles about three orthogonal axes. (Show all the important steps).}

\problemAnswer{
Solution.
}

\end{homeworkProblem}

%----------------------------------------------------------------------------
%	PROBLEM 9
%----------------------------------------------------------------------------

\begin{homeworkProblem}
\textit{There is no Problem 9. This is an example problem that demonstrates a  MATLAB listing, which is shown in Listing \ref{square}.}

\matlabscript{square}{Sample MATLAB script with syntax highlighting.}

\end{homeworkProblem}


%----------------------------------------------------------------------------

\end{document}